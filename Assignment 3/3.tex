
\documentclass[a4paper,12pt]{article}
\usepackage{amsmath}
\usepackage{fullpage}
\usepackage{enumitem}
\setlength{\parindent}{0pt}
\setlength{\parskip}{1em}
\usepackage{xcolor}
\usepackage{fancyhdr}
\usepackage{graphicx}
\usepackage{wrapfig}
\pagestyle{fancy}{}
\fancyhf{} % clear all fields

% Header: image left, text center, chapter right
\fancyhead[L]{\includegraphics[height=1cm]{IIIT.png}} % Replace logo.png with your image file
\fancyhead[C]{\textbf{Assignment 3}}
\fancyhead[R]{\textit{Name:SIVA CHETAN BALA \\ ID NO: fwc comet023}}

% Footer: page number in center
\fancyfoot[C]{\thepage}

% Optional: horizontal line under header and above footer
\renewcommand{\headrulewidth}{0.4pt}



\begin{document}
\thispagestyle{fancy}
\vspace*{2em}
\noindent\textbf{\textcolor{cyan}{Example 2:}}
Which of the following list of numbers form an AP? If they form an AP, write the next two terms : \\
(i) 4, 10, 16, 22, \ldots \hspace{1em}
(ii) 1, –1, –3, –5, \ldots \\
(iii) 2, –2, 2, –2, 2, –2, \ldots \hspace{1em}
(iv) 1, 1, 1, 2, 2, 2, 3, 3, 3, \ldots

\noindent\textbf{\textcolor{cyan}{Solution:}} 
(i) We have 
\hspace{0.5cm}\(
a_2 - a_1 = 10 - 4 = 6
\)
\[
a_3 - a_2 = 16 - 10 = 6
\]
\[
a_4 - a_3 = 22 - 16 = 6
\]\\
i.e., \(a_{k+1} - a_k\) is the same every time. \\
So, the given list of numbers forms an  AP with the common difference d = 6.\\
The next two terms are: 22 + 6 = 28  and  28 + 6 = 34.

(ii)
\[
a_2 - a_1 = -1 - 1 = -2
\]
\[
a_3 - a_2 = -3 - (-1) = -3 + 1 = -2
\]
\[
a_4 - a_3 = -5 - (-3) = -5 + 3 = -2
\]
i.e., \(a_{k+1} - a_k\) is the same every time. \\
So,the given list of numbers forms  an AP with the common difference d = -2. \\
The next two terms are: \\
\[
-5 + (-2) = -7 \quad \text{and} \quad -7 + (-2) = -9
\]

(iii)
\[
a_2 - a_1 = -2 - (2) = -2 - 2 = -4
\]
\[
a_3 - a_2 = 2 - (-2) = 2 + 2 = 4
\]
As \(a_2 - a_1 \ne a_3 - a_2\), the given list of numbers does not form an AP.

(iv)
\[
a_2 - a_1 = 1 - 1 = 0
\]
\[
a_3 - a_2 = 1 - 1 = 0
\]
\[
a_4 - a_3 = 2 - 1 = 1
\]
Here,\( a_2 - a_1 = a_3 - a_2 \ne a_4 - a_3\). \\
So, the given list of numbers does not form an AP.

\pagestyle{empty}
\vspace{5em}
\noindent\textbf{\textcolor{cyan}{Example 3:}}
Find the 10th term of the AP : 2, 7, 12, \ldots \\
\noindent\textbf{\textcolor{cyan}{Solution:}} 
Here, \(a = 2\), \quad d = 7 - 2 = 5 \quad \text{and} \quad\( n = 10\). \\
\vspace{2em}
We have 
\hspace{2cm}\(
          a_n = a + (n - 1)d
\) \\
So,
\hspace{3cm}\(
a_{10} = 2 + (10 - 1) \cdot 5 = 2 + 45 = 47
\) \\
Therefore, the 10th term of the given AP is 47.

\vspace{0.5em}
\noindent\textbf{\textcolor{cyan}{Example 4:}}
Which term of the AP: 21, 18, 15, \ldots is –81? Also, is any term 0? Give  reason for your answer. \\
\noindent\textbf{\textcolor{cyan}{Solution:}}\\  
Here, \(a = 21, \quad d = 18 - 21 = -3, \quad \text{and} \quad a_n = -81\), and we have to find \(n\).

\noindent\textbf{As:}\hspace{5.3cm} \(a_n = a + (n - 1)d\)

\noindent\textbf{We have:}
\begin{align*}
-81 &= 21 + (n - 1)(-3) \\
-81 &= 21 - 3(n - 1) \\
-81 &= 24 - 3n \\
-105 &= -3n \\
\Rightarrow n &= 35
\end{align*}

\noindent\textbf{Therefore,} the 35th term of the given AP is –81.

\noindent\textbf{Next,} we want to know if there is any \(n\) for which \(a_n = 0\). If such an \(n\) is there, then:
\begin{align*}
21 + (n - 1)(-3) &= 0 \\
3(n - 1) &= 21 \\
n &= 8
\end{align*}

\noindent\textbf{So,} the eighth term is 0.

\end{document}
