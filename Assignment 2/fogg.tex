\documentclass{article}
\usepackage{amsmath}
\usepackage{amssymb}
\usepackage{fancyhdr} % For headers and footers
\usepackage[margin=1in]{geometry}
\usepackage{graphicx}

% Header setup
\pagestyle{fancy}
\fancyhf{}
\fancyhead[L]{\includegraphics[width=0.2\textwidth]{iiit.png}}
\fancyhead[C]{Assignment 2}
\fancyhead[R]{Name:SIVA CHETAN BALA \\ID NO:
fwc comet023}
\fancyfoot[C]{IIIT Bangalore}

\begin{document}

\noindent\textbf{Q.36} 
{If \( X = 1 \) in the logic equation\[\left[ X + Z \left\{ \overline{Y} + \left( \overline{Z} + X \overline{Y} \right) \right\} \right]
\left\{ \overline{X} + \overline{Z}(X + Y) \right\} = 1,
\]then}

\vspace{0.5em}

\noindent
\begin{tabular}{llll}
(A) \( Y = Z \) & (B) \( Y = \overline{Z} \) & (C) \( Z = 1 \) & (D) \( Z = 0 \)
\end{tabular}

\vspace{1em}
\noindent\textbf{Solution:}

Given:
\[
\left[X + Z\{Y + (Z + XY)\}\right]\left\{\overline{X} + \overline{Z}(X + Y)\right\} = 1
\]

Substitute \( X = 1 \):

\begin{align*}
&\left[1 + Z\{Y + (Z + 1 \cdot Y)\}\right]\left\{\overline{1} + \overline{Z}(1 + Y)\right\} = 1 \\
&\left[1 + Z\{Y + Z + Y\}\right]\left\{0 + \overline{Z}(1 + Y)\right\} = 1 \\
&\left[1 + Z(2Y + Z)\right]\left[\overline{Z}(1 + Y)\right] = 1
\end{align*}

For the product to be 1, both factors must be non-zero.

\noindent The second factor is \(\overline{Z}(1 + Y)\). For this to be non-zero:
\[
\overline{Z} = 1 \Rightarrow Z = 0
\]

Now substitute \( Z = 0 \) into the full expression:

\begin{align*}
&\left[1 + 0\cdot(2Y + 0)\right](1 \cdot (1 + Y)) = 1 \\
&[1](1 + Y) = 1
\end{align*}

\noindent Since \( (1 + Y) \geq 1 \), the expression equals 1 regardless of \( Y \)'s value.

\noindent\textbf{Therefore, the correct answer is (D) } \( Z = 0 \).

\end{document}
